\newpage

\hypertarget{refs}{}
\hypertarget{ref-alston2018}{}
Alston, Julian M, James T Lapsley, and Olena Sambucci. 2018. ``Grape and
Wine Production in California.'' In \emph{California Agriculture:
Dimensions and Issues}, 1--28.

\begin{itemize}
\tightlist
  \item
    An overview of wine and grape production in California, including brief history highlighting important dates for wine.
  \item
    Not entirely focused on wine grapes. Includes information on area, volume, and value of production. Suggests that while raisin and table grape growth hasn't changed much or declined since the 1960s, wine grape production has increased dramatically. 
  \item
    Higlights the four biggest regions for viticulture in California: North Coast, Central Coast, North San Joaquin Valley, and South San Joaquin Valley
  \item
    Discusses the production strategies of each of the regions. Wine production in San Joaquin valley typically on a larger scale (by farm area) then North and Central coast regions. However, the region with the most individual farms is the North Coast. North and Central Coast regions tend to rely on less mechanized modes of production, lending themselves to smaller operations. >80\% of grape harvesting is done by machine in CA.
  \item
    What separates the four regions? "Terrain, climate, soil types, mixture of varieties grown, and quality of grapes and wine produced".
\end{itemize}

\hypertarget{ref-cahill2007}{}
Cahill, Kimberly Nicholas, David B Lobell, Christopher B Field, Celine
Bonfils, and Katharine Hayhoe. 2007. ``Modeling Climate and Climate
Change Impacts on Wine Grape Yields in California.'' \emph{American
Journal of Enology and Viticulture} 58 (3): 414A--414A.

\begin{itemize}
\tightlist
  \item
    Over 90\% of the wine in the United States comes from California, making CA the 4th largest producer of wine in the Uniter States
  \item
    Wine grapes are the most valuable fruit crop in the United States
  \item
    Authors used climate models to predict wine quality and production yields under two emissions scenarios. 
  \item
    Climate models used to project wine quality included the low-sensitivity PCM and medium-sensitivity HadCM3 model for two emission scenarios. Six different climate models were used to project yield estimates.
  \item
    Two quality metrics were projected based on estimates of future temperatures: ripening time (based on growing degree days, GDD) and average monthly temperatures over the derived ripening time.
  \item
    Historical linear regression models also fit to estimate role of climate on yields. Models were expanded to multiple regression by including the strongest predictors from first regression analysis. Optimal model was quadratic multiple regression considering monthly average minimum temperatues and precipitation in June, September, and April.
  \item
    Results suggest a plateau near the long-term mean (LTM) of historical yields as rainfall increases during June and the previous September. Similar pattern for April minimum temps.
  \item
    Projected changes to yields have large uncertainty, but estimate a decline below LTM.
  \item
    Projected changes to quality suggest the Central Valley being quality impaired by 2100 under both emission scenarios. Central and North Coast projected to remain near optimal conditions until late in century.
    
\end{itemize}

\hypertarget{ref-jones2007}{}
Jones, Gregory V. 2007. ``Climate Change: Observations, Projections, and
General Implications for Viticulture and Wine Production.''
\emph{Economics Department-Working Paper} 7: 14.

\begin{itemize}
\tightlist
  \item
    Viticulture typically exists (or is optimal) within narrow climatological bounds. Given these narrow bounds, short-term variation can have large effects on wine quality. Long-term changes can influence habitat suitability for specific wine grapes.
  \item
    A variety of metrics have been used to measure the impact of climate on wine: GDD, mean temperatures of the warmest month and growing season.
  \item
    Temperature thresholds exist where certain grape varietals lose suitability for planting in a region. Jones 2008 describes these in terms of average growing season temperatures. These temperatures are shifting through time.
  \item
    Flavor balance will be a good indicator for when a wine is passed a climate threshold
  \item
    Trends in increasing alcohol levels are related to climate variability and change
  \item
    Generally there have been trends of increasing wine quality and less vintage-to-vintage variation in quality. The relationships between climate and these variables is not always linear or uniform. 
  \item
    Early projections of wine quality increases due to warming climate in Bordeaux and Champagne have been shown to be accurate.
  \item
    Under projected warming for the next century, suitable habitat for viticulture is predicted to move upwards (i.e. elevation) and towards the coast in California. Habitat is projected to expand in Washington and Oregon by ~25%. 
  
\end{itemize}

\hypertarget{ref-jones2008}{}
Jones, Gregory V, and Gregory B Goodrich. 2008. ``Influence of Climate
Variability on Wine Regions in the Western Usa and on Wine Quality in
the Napa Valley.'' \emph{Climate Research} 35 (3): 241--54.

\begin{itemize}
\tightlist
  \item
    Wine growers typically associate El Niño with poor grape quality, but no research has actually been done to estimate the relationship in California. 
  \item
    The Pacific Decadal Oscillation has been shown to affect wine quality in California, but only in wines of Napa and Sonoma Counties
  \item
    Goal of this work was to understand the role of the PDO in climatic variables used in viticulture. Specifically the multi-decadal PDO and interannual ENSO
  \item
    Used United States Historical Climate Network data; 22 stations for California. Gave 10 regions to modeling climate variables through time between 1930-2002.
  \item
    A series of climate parameters were derived from the USHCN: 5 related to temperature/heat accumulation, 6 to frost incidence, and 4 to precipitation
  \item
    Years of El Niño and La Niña were identified by taking the 5 year moving average of SST within the "Niño-3.4" region, defined by a bounding box over the equatorial central Pacific. If the SST anomaly exceeded 0.4 \textdegree C for > 6 mo, then there was an El Niño. In the period selected, there were 19 El Niño years and 17 La Niña years. 30 of the years were neutral.
  \item
    The PDO index is the "leading principal component or eigenvector of the mean monthly SSTs in the Pacific Ocean north of 20 \textdegree N latitude." The PDO occurs on about a 50 year period, and results in warmer SSTs on on the west coast when in a positive phase.
  \item
    Less is known about the PDO than ENSO. The PDO is debatably ENSO "red noise" or the "superposition of SST fluctuations emanating from dynamical modes such as ENSO and the Kuroshio-Oyashio Extension".
  \item
    The PDO may be the cause of the step change in GDD between 1976-1977 in California
  \item
    Jones 2005 showed that a 1 \textdegree C increase in temperature increased wine ratings by 13 points
  \item
    Paper focuses specifically on Cabernet Sauvignons from Napa.
  \item
    Looking at the rolling average of the Wine Spectator ratings, my guess is that these ratings don't follow a normal distribution. I should investigate distributions that better fit the data.
  \item
    Wine ratings are highly correlated between sources (R^2 = 0.9). It's probably okay to rely on Wine Enthusiast ratings, unless more data are easily available.
  \item
    Average growing season temps increased by 0.9 \textdegree C between 1933 - 2002. Further, the number of days below 0 dropped by 29 days annually over the time period. 
  \item
    There were large changes in start of spring and fall. The last day of spring came 44 days earlier and first fall 41 days later.
  \item
    ENSO phase was not a significant factor for wine production model. 
  \item
    PDO was highly significant predictor for temperature variables affecting wine production, mainly due to more cold days during cold phases of the PDO. 
  \item
    No significant changes in precipitation through the time period
  \item
    No autocorrelation in derived climatic time series (Durbin-Watson)
    
\end{itemize}

\hypertarget{ref-large2013}{}
Large, Scott I, Gavin Fay, Kevin D Friedland, and Jason S Link. 2013.
``Defining Trends and Thresholds in Responses of Ecological Indicators
to Fishing and Environmental Pressures.'' \emph{ICES Journal of Marine
Science} 70 (4). Oxford University Press: 755--67.

\begin{itemize}
\tightlist
  \item
    Single-species methods for fisheries management do not typically consider environmental or ecosystem information in assessments. Ecosystem-based fisheries management is the field in which ecosystem information is considered within the process of management, either in qualitative or quantitative fashion. 
  \item
    Before management action can be taken, the levels at which ecosystem pressures affect some index of interest must be established. 
  \item
    Many studies provide address the general levels at which ecosystem processes change, but do not address "tipping points" explicitly
  \item
    Cumulative sums, sequential t-tests, empirical fluctuation processes have been used to identify changes in mean or variance of time series
  \item
    Thresholds exist when a small change in a pressure variable leads to large changes in the response. Quantifiable with Generalized Additive Models (GAMs).
  \item
    GAMs follow the notation, Y = \alpha + s(X) + \epsilon. The smoothing term, s(X), is a cubic spline fit to segments of the response variable series, Y. The smoothed terms are connected at "knots" that are optimized to themselves be smooth. \alpha is the intercept and \epsilon is the error term, assumed to have been drawn from a normal distribution with mean 0 and variance \sigma^2.

\hypertarget{ref-large2015}{}
---------. 2015. ``Quantifying Patterns of Change in Marine Ecosystem
Response to Multiple Pressures.'' \emph{PLoS One} 10 (3). Public Library
of Science: e0119922.

\begin{itemize}
\tightlist
  \item
    I included this paper as another reference for ecosystem threshold analyses. This paper uses gradient forest methods to quantify multivariate pressure-response relationships. I did not include any information from this work in my proposal.  
\end{itemize}


\hypertarget{ref-ncei2019}{}
NCEI, NOAA. 2019. ``Pacific Decadal Oscillation (Pdo).'' National
Oceanic; Atmospheric Administration.
\url{https://www.ncdc.noaa.gov/teleconnections/pdo/}.

\begin{itemize}
\tightlist
  \item
    This NOAA site provided insight into what the PDO is and how it is measured. It also links out to the PDO index time series.
\end{itemize}

\hypertarget{ref-salmon2016}{}
Salmon, Maëlle. 2016. \emph{Riem: Accesses Weather Data from the Iowa
Environment Mesonet}. \url{https://CRAN.R-project.org/package=riem}.

\begin{itemize}
\tightlist
  \item
    The riem R package provides access to ASOS sensor data via the Iowa Environment Mesonet.
\end{itemize}

\hypertarget{ref-samhouri2009}{}
Samhouri, Jameal F, Phillip S Levin, and Chris J Harvey. 2009.
``Quantitative Evaluation of Marine Ecosystem Indicator Performance
Using Food Web Models.'' \emph{Ecosystems} 12 (8). Springer: 1283--98.

\begin{itemize}
\tightlist
  \item
    This work explores different types of ecosystem thresholds and how they may be understood from a management perspective.
\end{itemize}

\hypertarget{ref-white2006}{}
White, Michael A, NS Diffenbaugh, Gregory V Jones, JS Pal, and F Giorgi.
2006. ``Extreme Heat Reduces and Shifts United States Premium Wine
Production in the 21st Century.'' \emph{Proceedings of the National
Academy of Sciences} 103 (30). National Acad Sciences: 11217--22.

\begin{itemize}
\tightlist
  \item
    Agricultural systems are vulnerable to the warming of the 21st century, and grapes are an optimal case for studying the effects of warming
  \item
    Grape production in the United States is valued at $2.9 billion (2006). 
  \item 
    Authors explored the effects of greenhouse gas-forced climated change models on a fine scale (25 km).
  \item
    Important to highlight the spatial heterogeneity at which climate change plays out
  \item
    Results project a shrinking and shifting of premium viticultural area in the United States.   \item
    Southwest and midwest United States projected to see the viticulture eliminated, save for high elevations.
  \item
    Considering mean climate alone without considering variability could hugely alter climate change impacts. If extreme temperatures are ignored during simulation runs, premium viticultural area is only marginally effected. When considering variability, the "marginal", or mid-quality areas are virtually eliminated from projections. 
  \item
    Changes in extreme heat were a much larger factor than changes to season phenology
  \item
    Projections suggest a shift towards higher humidity regions, which tend to coincide with other issues for viticultue, such as fungal infections. Breeding programs and technological innovations may be helpful for addressing these changes. 
    
\end{itemize}

\hypertarget{ref-wickham2019}{}
Wickham, Hadley. 2019. \emph{Rvest: Easily Harvest (Scrape) Web Pages}.
\url{https://CRAN.R-project.org/package=rvest}.

\begin{itemize}
\tightlist
  \item
    The rvest R package provides a simple set of functions for navigating web scraping using R. I used this package to scrape Wine Enthusiast ratings data.
\end{itemize}

\end{document}
