\newpage

\hypertarget{refs}{}
\hypertarget{ref-alston2018}{}
Alston, Julian M, James T Lapsley, and Olena Sambucci. 2018. ``Grape and
Wine Production in California.'' In \emph{California Agriculture:
Dimensions and Issues}, 1--28.

\begin{itemize}
\tightlist
  \item
    An overview of wine and grape production in California, including brief history highlighting important dates for wine.
  \item
    Not entirely focused on wine grapes. Includes information on area, volume, and value of production. Suggests that while raisin and table grape growth hasn't changed much or declined since the 1960s, wine grape production has increased dramatically. 
  \item
    Higlights the four biggest regions for viticulture in California: North Coast, Central Coast, North San Joaquin Valley, and South San Joaquin Valley
  \item
    Discusses the production strategies of each of the regions. Wine production in San Joaquin valley typically on a larger scale (by farm area) then North and Central coast regions. However, the region with the most individual farms is the North Coast. North and Central Coast regions tend to rely on less mechanized modes of production, lending themselves to smaller operations. >80\% of grape harvesting is done by machine in CA.
  \item
    What separates the four regions? "Terrain, climate, soil types, mixture of varieties grown, and quality of grapes and wine produced".
\end{itemize}

\hypertarget{ref-cahill2007}{}
Cahill, Kimberly Nicholas, David B Lobell, Christopher B Field, Celine
Bonfils, and Katharine Hayhoe. 2007. ``Modeling Climate and Climate
Change Impacts on Wine Grape Yields in California.'' \emph{American
Journal of Enology and Viticulture} 58 (3): 414A--414A.

\begin{itemize}
\tightlist
  \item
    Over 90\% of the wine in the United States comes from California, making CA the 4th largest producer of wine in the Uniter States
  \item
    Wine grapes are the most valuable fruit crop in the United States
  \item
    Authors used climate models to predict wine quality and production yields under two emissions scenarios. 
  \item
    Climate models used to project wine quality included the low-sensitivity PCM and medium-sensitivity HadCM3 model for two emission scenarios. Six different climate models were used to project yield estimates.
  \item
    Two quality metrics were projected based on estimates of future temperatures: ripening time (based on growing degree days, GDD) and average monthly temperatures over the derived ripening time.
  \item
    Historical linear regression models also fit to estimate role of climate on yields. Models were expanded to multiple regression by including the strongest predictors from first regression analysis. Optimal model was quadratic multiple regression considering monthly average minimum temperatues and precipitation in June, September, and April.
  \item
    Results suggest a plateau near the long-term mean (LTM) of historical yields as rainfall increases during June and the previous September. Similar pattern for April minimum temps.
  \item
    Projected changes to yields have large uncertainty, but estimate a decline below LTM.
  \item
    Projected changes to quality suggest the Central Valley being quality impaired by 2100 under both emission scenarios. Central and North Coast projected to remain near optimal conditions until late in century.
    
\end{itemize}

\hypertarget{ref-jones2007}{}
Jones, Gregory V. 2007. ``Climate Change: Observations, Projections, and
General Implications for Viticulture and Wine Production.''
\emph{Economics Department-Working Paper} 7: 14.

\hypertarget{ref-jones2008}{}
Jones, Gregory V, and Gregory B Goodrich. 2008. ``Influence of Climate
Variability on Wine Regions in the Western Usa and on Wine Quality in
the Napa Valley.'' \emph{Climate Research} 35 (3): 241--54.

\hypertarget{ref-large2013}{}
Large, Scott I, Gavin Fay, Kevin D Friedland, and Jason S Link. 2013.
``Defining Trends and Thresholds in Responses of Ecological Indicators
to Fishing and Environmental Pressures.'' \emph{ICES Journal of Marine
Science} 70 (4). Oxford University Press: 755--67.

\hypertarget{ref-large2015}{}
---------. 2015. ``Quantifying Patterns of Change in Marine Ecosystem
Response to Multiple Pressures.'' \emph{PLoS One} 10 (3). Public Library
of Science: e0119922.

\hypertarget{ref-ncei2019}{}
NCEI, NOAA. 2019. ``Pacific Decadal Oscillation (Pdo).'' National
Oceanic; Atmospheric Administration.
\url{https://www.ncdc.noaa.gov/teleconnections/pdo/}.

\hypertarget{ref-salmon2016}{}
Salmon, Maëlle. 2016. \emph{Riem: Accesses Weather Data from the Iowa
Environment Mesonet}. \url{https://CRAN.R-project.org/package=riem}.

\hypertarget{ref-samhouri2009}{}
Samhouri, Jameal F, Phillip S Levin, and Chris J Harvey. 2009.
``Quantitative Evaluation of Marine Ecosystem Indicator Performance
Using Food Web Models.'' \emph{Ecosystems} 12 (8). Springer: 1283--98.

\hypertarget{ref-white2006}{}
White, Michael A, NS Diffenbaugh, Gregory V Jones, JS Pal, and F Giorgi.
2006. ``Extreme Heat Reduces and Shifts United States Premium Wine
Production in the 21st Century.'' \emph{Proceedings of the National
Academy of Sciences} 103 (30). National Acad Sciences: 11217--22.

\hypertarget{ref-wickham2019}{}
Wickham, Hadley. 2019. \emph{Rvest: Easily Harvest (Scrape) Web Pages}.
\url{https://CRAN.R-project.org/package=rvest}.


\end{document}
